\documentclass[fleqn, a4paper, 11pt, oneside]{amsart}
%\usepackage[top = 2cm, bottom = 1cm, left = 1cm, right = 1cm]{geometry}
\usepackage{exsheets, tasks}
\usepackage{amsmath, amssymb, amsthm} %standard AMS packages
\usepackage{marginnote} %marginnotes
\usepackage{gensymb} %miscellaneous symbols
\usepackage{commath} %differential symbols
\usepackage{xcolor} %colours
\usepackage{cancel} %cancelling terms
\usepackage[free-standing-units, space-before-unit]{siunitx} %formatting units
\usepackage{tikz, pgfplots} %diagrams
\usetikzlibrary{calc, hobby, patterns, intersections, decorations.markings}
\usepackage{graphicx} %inserting graphics
\usepackage{hyperref} %hyperlinks
\usepackage{datetime} %date and time
\usepackage{ulem} %underline for \emph{}
\usepackage{xfrac} %inline fractions
\usepackage{enumerate,enumitem} %numbered lists
\usepackage{float} %inserting floats
\usepackage{circuitikz}[american voltages, american currents] %circuit diagrams

\newcommand\numberthis{\addtocounter{equation}{1}\tag{\theequation}} %adds numbers to specific equations in non-numbered list of equations

\newcommand{\AxisRotator}[1][rotate=0]{
	\tikz [x=0.25cm,y=0.60cm,line width=.2ex,-stealth,#1] \draw (0,0) arc (-150:150:1 and 1);%
} %rotation symbols on axes

\theoremstyle{definition}
\newtheorem{example}{Example}
\newtheorem{definition}{Definition}

\theoremstyle{theorem}
\newtheorem{theorem}{Theorem}

\newcommand{\curl}{\mathrm{curl\,}}

\newcommand{\Arg}{\mathrm{Arg}}

\newcommand{\Int}{\mathrm{Int}}
\newcommand{\Ext}{\mathrm{Ext}}
\newcommand{\boundary}{\partial}

\makeatletter
\@addtoreset{section}{part} %resets section numbers in new part
\makeatother

\renewcommand{\thesubsection}{(\arabic{subsection})}
\renewcommand{\thesection}{(\arabic{section})}

%section headings on left
\makeatletter
\def\specialsection{\@startsection{section}{1}%
	\z@{\linespacing\@plus\linespacing}{.5\linespacing}%
	%  {\normalfont\centering}}% DELETED
	{\normalfont}}% NEW
\def\section{\@startsection{section}{1}%
	\z@{.7\linespacing\@plus\linespacing}{.5\linespacing}%
	%  {\normalfont\scshape\centering}}% DELETED
	{\normalfont\scshape}}% NEW
\makeatother

%forces newline after subsection
\makeatletter
\def\subsection{\@startsection{subsection}{3}%
	\z@{.5\linespacing\@plus.7\linespacing}{.1\linespacing}%
	{\normalfont\itshape}}
\makeatother

\settasks{counter-format = tsk[1].}

\SetupExSheets{solution/print = true}

%opening
\title{Complex Functions : Assignment 3}
\author
{
	Aakash Jog\\
	ID : 989323563
}
\date{\formatdate{18}{11}{2015}}

\begin{document}

\tikzset{->-/.style={decoration={
  markings,
  mark=at position #1 with {\arrow{>}}},postaction={decorate}}}

\maketitle
%\setlength{\mathindent}{0pt}

\setcounter{question}{0}
\begin{question}
	Prove or disprove.
	\begin{enumerate}
		\item $\lim\limits_{z \to 0} e^{\frac{1}{z}} = \infty$
		\item $\lim\limits_{z \to 0} e^{\frac{1}{|z|}} = \infty$
	\end{enumerate}
\end{question}

\begin{solution}
	\begin{enumerate}[leftmargin=*]
		\item
			\begin{align*}
				\lim\limits_{z \to 0} e^{\frac{1}{z}} & = \lim\limits_{r \to 0} e^{\frac{1}{r e^{i \theta}}}               \\
                                                                      & = \lim\limits_{r \to 0} {e^{\frac{1}{r}}}^{\frac{1}{e^{i \theta}}} \\
                                                                      & = \infty
			\end{align*}
		\item
			\begin{align*}
				\lim\limits_{z \to 0} e^{\frac{1}{|z|}} & = \lim\limits_{r \to 0} e^{\frac{1}{r}} \\
                                                                        & = \infty
			\end{align*}
	\end{enumerate}
\end{solution}

\setcounter{question}{1}
\begin{question}
	Find the domain of analyticity of the following functions.
	\begin{enumerate}
		\setcounter{enumi}{1}
		\item $f(z) = \frac{z^2 + 1}{(z + 2) \left( z^2 + 2 z + 2 \right)}$
	\end{enumerate}
\end{question}

\begin{solution}
	\begin{enumerate}[leftmargin=*]
		\setcounter{enumi}{1}
		\item
			\begin{align*}
				f(z) & = \frac{z^2 + 1}{(z + 2) \left( z^2 + 2 z + 2 \right)} \\
                                     & = \frac{z^2 + 1}{(z + 2) (z + 1 - i) (z + 1 + i)}
			\end{align*}
			Therefore, $f(z)$ is defined and is differentiable over $\mathbb{C} \setminus \{-2 , -1 + i , -1 - i\}$.\\
			Hence, $f(z)$ is analytic at $(-2,0)$, $(-1,1)$, $(-1,-1)$.
	\end{enumerate}
\end{solution}

\setcounter{question}{3}
\begin{question}
	Calculate $f'(z)$.
	\begin{enumerate}
		\setcounter{enumi}{1}
		\item $f(z) = \frac{8 z^2 - 3}{z^2 + 1}$
	\end{enumerate}
\end{question}

\begin{solution}
	\begin{enumerate}
		\setcounter{enumi}{1}
		\item
			\begin{align*}
				f(z) & = \frac{8 z^2 - 3}{z^2 + 1}      \\
                                     & = \frac{8 z^2 + 8 - 11}{z^2 + 1} \\
                                     & = 8 - \frac{11}{z^2 + 1}
			\end{align*}
			Therefore, as $f(z)$ is defined on $\mathbb{C} \setminus \{(0,1),(0,-1)\}$, it is analytic at $(0,1)$ and $(0,-1)$.
			Therefore,
			\begin{align*}
				f'(z) & = \frac{22}{\left( z^2 + 1 \right)^2}
			\end{align*}
	\end{enumerate}
\end{solution}

\setcounter{question}{5}
\begin{question}
	Let $f(z) = \overline{z}$.
	Show that $f$ doesn't satisfy the polar Cauchy-Riemann equations and conclude that $f$ isn't differentiable at any point in the plane.
\end{question}

\begin{solution}
	\begin{align*}
		f(z) & = \overline{z} \\
                     & = x - i y
	\end{align*}
	Therefore,
	\begin{align*}
		u(r,\theta) & = x             \\
                            & = r \cos \theta \\
		v(r,\theta) & = -y            \\
                            & = -r \sin \theta
	\end{align*}
	Therefore,
	\begin{align*}
		u_r        & = \cos \theta    \\
		u_{\theta} & = -r \sin \theta \\
		v_r        & = -\sin \theta   \\
		v_{\theta} & = -r \cos \theta
	\end{align*}
	Therefore,
	\begin{align*}
		u_r                 & = v_{\theta}     \\
		\iff \cos \theta    & = -r \cos \theta \\
		\iff r              & = -1             \\
		u_{\theta}          & = -v_r           \\
		\iff -r \sin \theta & = \sin \theta    \\
		\iff r              & = -1
	\end{align*}
	However, $r$ cannot be negative.
	Therefore, the function does not satisfy the Cauchy-Riemann equations, and hence is not differentiable at any point in the plane.
\end{solution}

\end{document}
