\documentclass[fleqn, a4paper, 11pt, oneside]{amsart}
%\usepackage[top = 2cm, bottom = 1cm, left = 1cm, right = 1cm]{geometry}
\usepackage{exsheets, tasks}
\usepackage{amsmath, amssymb, amsthm} %standard AMS packages
\usepackage{marginnote} %marginnotes
\usepackage{gensymb} %miscellaneous symbols
\usepackage{commath} %differential symbols
\usepackage{xcolor} %colours
\usepackage{cancel} %cancelling terms
\usepackage[free-standing-units, space-before-unit]{siunitx} %formatting units
\usepackage{tikz, pgfplots} %diagrams
\usetikzlibrary{calc, hobby, patterns, intersections, decorations.markings}
\usepackage{graphicx} %inserting graphics
\usepackage{hyperref} %hyperlinks
\usepackage{datetime} %date and time
\usepackage{ulem} %underline for \emph{}
\usepackage{xfrac} %inline fractions
\usepackage{enumerate,enumitem} %numbered lists
\usepackage{float} %inserting floats
\usepackage{circuitikz}[american voltages, american currents] %circuit diagrams

\newcommand\numberthis{\addtocounter{equation}{1}\tag{\theequation}} %adds numbers to specific equations in non-numbered list of equations

\newcommand{\AxisRotator}[1][rotate=0]{
	\tikz [x=0.25cm,y=0.60cm,line width=.2ex,-stealth,#1] \draw (0,0) arc (-150:150:1 and 1);%
} %rotation symbols on axes

\theoremstyle{definition}
\newtheorem{example}{Example}
\newtheorem{definition}{Definition}

\theoremstyle{theorem}
\newtheorem{theorem}{Theorem}

\newcommand{\curl}{\mathrm{curl\,}}

\makeatletter
\@addtoreset{section}{part} %resets section numbers in new part
\makeatother

\renewcommand{\thesubsection}{(\arabic{subsection})}
\renewcommand{\thesection}{(\arabic{section})}

%section headings on left
\makeatletter
\def\specialsection{\@startsection{section}{1}%
	\z@{\linespacing\@plus\linespacing}{.5\linespacing}%
	%  {\normalfont\centering}}% DELETED
	{\normalfont}}% NEW
\def\section{\@startsection{section}{1}%
	\z@{.7\linespacing\@plus\linespacing}{.5\linespacing}%
	%  {\normalfont\scshape\centering}}% DELETED
	{\normalfont\scshape}}% NEW
\makeatother

%forces newline after subsection
\makeatletter
\def\subsection{\@startsection{subsection}{3}%
	\z@{.5\linespacing\@plus.7\linespacing}{.1\linespacing}%
	{\normalfont\itshape}}
\makeatother

\settasks{counter-format = tsk[1].}

\SetupExSheets{solution/print = true}

%opening
\title{Complex Functions : Assignment 1}
\author
{
	Aakash Jog\\
	ID : 989323563
}
\date{\formatdate{28}{10}{2015}}

\begin{document}

\tikzset{->-/.style={decoration={
  markings,
  mark=at position #1 with {\arrow{>}}},postaction={decorate}}}

\maketitle
%\setlength{\mathindent}{0pt}

\setcounter{question}{0}
\begin{question}
	Calculate
	\begin{enumerate}
		\item $\frac{i}{1 - i} + \frac{1 - i}{i}$
		\item $\left| \frac{1 + 3 i}{3 + i} \right|$
	\end{enumerate}
\end{question}

\begin{solution}
	\begin{enumerate}[leftmargin=*]
		\item
			\begin{align*}
				\frac{i}{1 - i} + \frac{1 - i}{i} & = \frac{i^2}{i (1 - i)} + \frac{(1 - i)^2}{i (1 - i)} \\
                                                                  & = \frac{-1 + 1 - 2 i - 1}{i + 1}                      \\
                                                                  & = -\frac{2 i + 1}{i + 1}                              \\
                                                                  & = -\frac{(2 i + 1) (i - 1)}{-1 - 1}                   \\
                                                                  & = -\frac{-2 - 2 i + i - 1}{-2}                        \\
                                                                  & = -\frac{-3 - i}{-2}                                  \\
                                                                  & = \frac{-3 - i}{2}
			\end{align*}
		\item
			\begin{align*}
				\left| \frac{1 + 3 i}{3 + i} \right| & = \left| \frac{(1 + 3 i) (3 - i)}{9 + 1} \right|                     \\
                                                                     & = \left| \frac{3 - i + 9 i + 3}{10} \right|                          \\
                                                                     & = \left| \frac{6 + 8 i}{10} \right|                                  \\
                                                                     & = \left| \frac{3}{5} + \frac{4}{5} i \right|                         \\
                                                                     & = \sqrt{\left( \frac{3}{5} \right)^2 + \left( \frac{4}{5} \right)^2} \\
                                                                     & = \sqrt{\frac{9 + 16}{25}}                                           \\
                                                                     & = 1
			\end{align*}
	\end{enumerate}
\end{solution}

\setcounter{question}{3}
\begin{question}
	Prove
	\begin{enumerate}
		\item $\overline{z} = \frac{1}{z} \iff |z| = 1$
		\item $z + \frac{1}{z} \in \mathbb{R} \iff |z| = 1 \text{ or } z \in \mathbb{R}$
	\end{enumerate}
\end{question}

\begin{solution}
	\begin{enumerate}[leftmargin=*]
		\item
			\begin{align*}
				\frac{1}{z} & = \frac{\overline{z}}{z \overline{z}} \\
                                            & = \frac{\overline{z}}{|z^2|}
			\end{align*}
			Therefore,
			\begin{align*}
				\frac{1}{z} & = \overline{z} \\
				\iff |z^2|  & = 1            \\
				\iff |z|    & = 1
			\end{align*}
			\qed
		\item
			Let
			\begin{align*}
				z & = x + iy
			\end{align*}
			Therefore,
			\begin{align*}
				z + \frac{1}{z} & = x + i y + \frac{1}{x + i y}                                 \\
                                                & = x + i y + \frac{x - i y}{x^2 + y^2}                         \\
                                                & = \frac{x (x^2 + y^2) + i y (x^2 + y^2) + x - i y}{x^2 + y^2} \\
                                                & = \frac{x^3 + x y^2 + i x^2 y + i y^3 + x - i y}{x^2 + y^2}   \\
                                                & = \frac{x^3 + x y^2 + x}{x^2 + y^2} + i \frac{x^2 y + y^3 - y}{x^2 + y^2}
			\end{align*}
			Therefore, $z + \frac{1}{z} \in \mathbb{R}$, if and only if
			\begin{align*}
				x^2 y + y^3 - y        & = 0 \\
				\iff y (x^2 + y^2 - 1) & = 0 \\
				\iff y (|z|^2 - 1)     & = 0
			\end{align*}
			If and only if
			\begin{align*}
				y      & = 0            & \text{ or } &  & |z|^2 - 1 & = 0 \\
				\iff z & \in \mathbb{R} & \text{ or } &  & |z|^2     & = 1 \\
				\iff z & \in \mathbb{R} & \text{ or } &  & |z|       & = 1
			\end{align*}
			\qed
	\end{enumerate}
\end{solution}

\setcounter{question}{4}
\begin{question}
	Write the following in polar coordinates and find the argument set $\arg(z)$.
	\begin{enumerate}
		\item $-4 i$
		\item $-2 + 2 i$
		\item $1 - i$
		\item $\frac{3 - 4 i}{2 - i}$
	\end{enumerate}
\end{question}

\begin{solution}
	\begin{enumerate}[leftmargin=*]
		\item
			\begin{align*}
				z & = -4 i
			\end{align*}
			Let
			\begin{align*}
				z & = x + i y
			\end{align*}
			Therefore,
			\begin{align*}
				x & = 0 \\
				y & = -4
			\end{align*}
			Therefore,
			\begin{align*}
				r      & = \sqrt{x^2 + y^2}                     \\
                                       & = \sqrt{0 + (-4)^2}                    \\
                                       & = 4                                    \\
				\theta & = \tan^{-1}\left( \frac{y}{x} \right)  \\
                                       & = \tan^{-1}\left( \frac{-4}{0} \right) \\
                                       & = -\frac{\pi}{2}
			\end{align*}
			Therefore,
			\begin{align*}
				-4 i & = \left( 4 , -\frac{\pi}{2} \right)
			\end{align*}
			The argument set is
			\begin{align*}
				\arg(z) & = \left\{ -\frac{\pi}{2} + 2 \pi k : k \in \mathbb{Z} \right\}
			\end{align*}
		\item
			\begin{align*}
				z & = -2 + 2 i
			\end{align*}
			Let
			\begin{align*}
				z & = x + i y
			\end{align*}
			Therefore,
			\begin{align*}
				x & = -2 \\
				y & = 2
			\end{align*}
			Therefore,
			\begin{align*}
				r      & = \sqrt{x^2 + y^2}                     \\
                                       & = \sqrt{4 + 4}                         \\
                                       & = 2 \sqrt{2}                           \\
				\theta & = \tan^{-1}\left( \frac{y}{x} \right)  \\
                                       & = \tan^{-1}\left( \frac{2}{-2} \right) \\
                                       & = \frac{3 \pi}{2}
			\end{align*}
			Therefore,
			\begin{align*}
				-2 + 2 i & = \left( 2 \sqrt{2} , \frac{3 \pi}{2} \right)
			\end{align*}
			The argument set is
			\begin{align*}
				\arg(z) & = \left\{ \frac{3 \pi}{2} + 2 \pi k : k \in \mathbb{Z} \right\}
			\end{align*}
		\item
			\begin{align*}
				z & = 1 - i
			\end{align*}
			Let
			\begin{align*}
				z & = x + i y
			\end{align*}
			Therefore,
			\begin{align*}
				x & = 1 \\
				y & = -1
			\end{align*}
			Therefore,
			\begin{align*}
				r      & = \sqrt{x^2 + y^2}                     \\
                                       & = \sqrt{1 + 1}                         \\
                                       & = \sqrt{2}                             \\
				\theta & = \tan^{-1}\left( \frac{y}{x} \right)  \\
                                       & = \tan^{-1}\left( \frac{-1}{1} \right) \\
                                       & = -\frac{\pi}{2}
			\end{align*}
			Therefore,
			\begin{align*}
				1 - i & = \left( \sqrt{2} , -\frac{\pi}{2} \right)
			\end{align*}
			The argument set is
			\begin{align*}
				\arg(z) & = \left\{ -\frac{\pi}{2} + 2 \pi k : k \in \mathbb{Z} \right\}
			\end{align*}
		\item
			Let
			\begin{align*}
				z_1 & = 3 - 4 i \\
				z_2 & = 2 - i
			\end{align*}
			Let
			\begin{align*}
				z_1 & = x_1 + i y_1 \\
				z_2 & = x_2 + i y_2
			\end{align*}
			Therefore,
			\begin{align*}
				x_1 & = 3  \\
				y_1 & = -4 \\
				x_2 & = 2  \\
				y_2 & = -1
			\end{align*}
			Therefore,
			\begin{align*}
				r_1      & = \sqrt{{x_1}^2 + {y_1}^2}                \\
                                         & = \sqrt{9 + 16}                           \\
                                         & = \sqrt{25}                               \\
                                         & = 5                                       \\
				\theta_1 & = \tan^{-1}\left( \frac{y_1}{x_1} \right) \\
                                         & = \tan^{-1}\left( \frac{-4}{3} \right)    \\
                                         & = -\tan^{-1}\left( \frac{4}{3} \right)    \\
				r_2      & = \sqrt{{x_2}^2 + {y_2}^2}                \\
                                         & = \sqrt{4 + 1}                            \\
                                         & = \sqrt{5}                                \\
				\theta_2 & = \tan^{-1}\left( \frac{y_2}{x_2} \right) \\
                                         & = \tan^{-1}\left( \frac{-1}{2} \right)    \\
                                         & = -\tan^{-1}\left( \frac{1}{2} \right)
			\end{align*}
			Therefore,
			\begin{align*}
				3 - 4 i & = \left( 5 , -\tan^{-1}\left( \frac{4}{3} \right) \right) \\
				2 - i   & = \left( \sqrt{5} , -\tan^{-1}\left( \frac{1}{2} \right) \right)
			\end{align*}
			Therefore,
			\begin{align*}
				\frac{z_1}{z_2}                  & = \left( \frac{r_1}{r_2} , \theta_1 - \theta_2 \right)                                                           \\
				\therefore \frac{3 - 4 i}{2 - i} & = \left( \frac{5}{\sqrt{5}} , -\tan^{-1}\left( \frac{4}{3} \right) + \tan^{-1}\left( \frac{1}{2} \right) \right) \\
                                                                 & = \left( \sqrt{5} , -\tan^{-1}\left( \frac{4}{3} \right) + \tan^{-1}\left( \frac{1}{2} \right) \right)
			\end{align*}
			The argument set is
			\begin{align*}
				\arg(z) & = \left\{ -\tan^{-1}\left( \frac{4}{3} \right) + \tan^{-1}\left( \frac{1}{2} \right) + 2 \pi k : k \in \mathbb{Z} \right\}
			\end{align*}
	\end{enumerate}
\end{solution}

\setcounter{question}{7}
\begin{question}
	Prove that for every $z_1,z_2 \in \mathbb{C}$
	\begin{align*}
		|z_1 + z_2| & \ge \left| |z_1| - |z_2| \right|
	\end{align*}
\end{question}

\begin{solution}
	\begin{align*}
		|z_1|                    & = |z_1 + z_2 - z_2|      \\
                                         & \le |z_1 + z_2| + |-z_2| \\
		\therefore |z_1|         & \le |z_1 + z_2| + |z_2|  \\
		\therefore |z_1| - |z_2| & \le |z_1 + z_2|          \\
		\therefore |z_1 + z_2|   & \ge \left| |z_1| - |z_2| \right|
	\end{align*}
\end{solution}

\end{document}
