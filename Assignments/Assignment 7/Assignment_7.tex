\documentclass[fleqn, a4paper, 11pt, oneside]{amsart}
%\usepackage[top = 2cm, bottom = 1cm, left = 1cm, right = 1cm]{geometry}
\usepackage{exsheets, tasks}
\usepackage{amsmath, amssymb, amsthm} %standard AMS packages
\usepackage{marginnote} %marginnotes
\usepackage{gensymb} %miscellaneous symbols
\usepackage{commath} %differential symbols
\usepackage{xcolor} %colours
\usepackage{cancel} %cancelling terms
\usepackage[free-standing-units, space-before-unit]{siunitx} %formatting units
\usepackage{tikz, pgfplots} %diagrams
\usetikzlibrary{calc, hobby, patterns, intersections, decorations.markings}
\usepackage{graphicx} %inserting graphics
\usepackage{hyperref} %hyperlinks
\usepackage{datetime} %date and time
\usepackage{ulem} %underline for \emph{}
\usepackage{xfrac} %inline fractions
\usepackage{enumerate,enumitem} %numbered lists
\usepackage{float} %inserting floats
\usepackage{circuitikz}[american voltages, american currents] %circuit diagrams

\newcommand\numberthis{\addtocounter{equation}{1}\tag{\theequation}} %adds numbers to specific equations in non-numbered list of equations

\newcommand{\AxisRotator}[1][rotate=0]{
	\tikz [x=0.25cm,y=0.60cm,line width=.2ex,-stealth,#1] \draw (0,0) arc (-150:150:1 and 1);%
} %rotation symbols on axes

\theoremstyle{definition}
\newtheorem{example}{Example}
\newtheorem{definition}{Definition}

\theoremstyle{theorem}
\newtheorem{theorem}{Theorem}

\renewcommand{\Re}{\mathrm{Re}}
\renewcommand{\Im}{\mathrm{Im}}

\DeclareMathOperator{\Arg}{Arg}

\DeclareMathOperator{\Int}{Int}
\DeclareMathOperator{\Ext}{Ext}
\DeclareMathOperator{\boundary}{\partial}

\DeclareMathOperator{\Log}{Log}
\DeclareMathOperator{\pv}{pv}

\DeclareMathOperator{\length}{length}

\makeatletter
\@addtoreset{section}{part} %resets section numbers in new part
\makeatother

\renewcommand{\thesubsection}{(\arabic{subsection})}
\renewcommand{\thesection}{(\arabic{section})}

%section headings on left
\makeatletter
\def\specialsection{\@startsection{section}{1}%
	\z@{\linespacing\@plus\linespacing}{.5\linespacing}%
	%  {\normalfont\centering}}% DELETED
	{\normalfont}}% NEW
\def\section{\@startsection{section}{1}%
	\z@{.7\linespacing\@plus\linespacing}{.5\linespacing}%
	%  {\normalfont\scshape\centering}}% DELETED
	{\normalfont\scshape}}% NEW
\makeatother

%forces newline after subsection
\makeatletter
\def\subsection{\@startsection{subsection}{3}%
	\z@{.5\linespacing\@plus.7\linespacing}{.1\linespacing}%
	{\normalfont\itshape}}
\makeatother

\settasks{counter-format = tsk[1].}

\SetupExSheets{solution/print = true}

%opening
\title{Complex Functions : Assignment 7}
\author
{
	Aakash Jog\\
	ID : 989323563
}
\date{\formatdate{16}{12}{2015}}

\begin{document}

\tikzset{->-/.style={decoration={
  markings,
  mark=at position #1 with {\arrow{>}}},postaction={decorate}}}

\maketitle
%\setlength{\mathindent}{0pt}

\setcounter{question}{0}
\begin{question}
	Calculate the length of the following curves using the correct parametrization.
	\begin{enumerate}
		\setcounter{enumi}{3}
		\item $\gamma(t) = (t - \sin t) + i (1 - \cos t)$, $t \in [0,2 \pi]$.
	\end{enumerate}
\end{question}

\begin{solution}
	\begin{enumerate}[leftmargin=*]
		\setcounter{enumi}{3}
		\item
			\begin{align*}
				\gamma(t)                  & = (t - \sin t) + i (1 - \cos t) \\
				\therefore \dot{\gamma}(t) & = 1 - \cos t + i \sin t
			\end{align*}
			Therefore,
			\begin{align*}
				\left| \dot{\gamma}(t) \right| & = \sqrt{(1 - \cos t)^2 + \sin^2 t}          \\
                                                               & = \sqrt{1 - 2 \cos t + \cos^2 t + \sin^2 t} \\
                                                               & = \sqrt{2 - 2 \cos t}
			\end{align*}
			Therefore,
			\begin{align*}
				\length(\gamma) & = \int\limits_{a}^{b} \left| \dot{\gamma}(t) \right| \dif t  \\
                                                & = \int\limits_{0}^{2 \pi} \sqrt{2 - 2 \cos t} \dif t         \\
                                                & = \int\limits_{0}^{2 \pi} \sqrt{4 \sin^2 \frac{t}{2}} \dif t \\
                                                & = \int\limits_{0}^{2 \pi} 2 \sin \frac{t}{2} \dif t          \\
                                                & = \left. -4 \cos \frac{t}{2} \right|_{0}^{2 \pi}             \\
                                                & = 8
			\end{align*}
	\end{enumerate}
\end{solution}

\setcounter{question}{1}
\begin{question}
	Calculate $\displaystyle \frac{1}{2 \pi} \int\limits_{0}^{2 \pi} e^{i m \theta} e^{-i n \theta} \dif \theta$, $m,n \in \mathbb{Z}$.
	Hint: Divide to the cases $m = n$, and $m \neq n$.
\end{question}

\begin{solution}
	If $m = n$,
	\begin{align*}
		\frac{1}{2 \pi} \int\limits_{0}^{2 \pi} e^{i m \theta} e^{-i n \theta} \dif \theta & = \frac{1}{2 \pi} \int\limits_{0}^{2 \pi} \dif \theta \\
                                                                                                   & = 1
	\end{align*}
	If $m \neq n$,
	\begin{align*}
		\frac{1}{2 \pi} \int\limits_{0}^{2 \pi} e^{i m \theta} e^{-i n \theta} \dif \theta & = \frac{1}{2 \pi} \int\limits_{0}^{2 \pi} e^{i (m - n) \theta} e^{i n \theta} e^{-i n \theta} \dif \theta \\
                                                                                                   & = \frac{1}{2 \pi} \int\limits_{0}^{2 \pi} e^{i (m - n) \theta} \dif \theta                                \\
                                                                                                   & = \frac{1}{2 \pi} \left. \frac{e^{i (m - n) \theta}}{i (m - n)} \right|_{0}^{2 \pi}                       \\
                                                                                                   & = 0
	\end{align*}
\end{solution}

\setcounter{question}{2}
\begin{question}
	Prove
	\begin{enumerate}
		\setcounter{enumi}{0}
		\item
			The function $f(z) = \overline{z}$ has no analytic primitive which is analytic at every point.
	\end{enumerate}
\end{question}

\begin{solution}
	\begin{enumerate}[leftmargin=*]
		\setcounter{enumi}{0}
		\item
			Let $\gamma_1$ be the line joining $0$ and $1$.\\
			Therefore,
			\begin{align*}
				\gamma_1(t)                  & = t \\
				\therefore \dot{\gamma_1}(t) & = 1
			\end{align*}
			where $t \in (0,1)$.\\
			Let $\gamma_2$ be the line joining $1$ and $1 + i$.\\
			Therefore,
			\begin{align*}
				\gamma_2(t)                  & = 1 + i t \\
				\therefore \dot{\gamma_2}(t) & = i
			\end{align*}
			where $t \in (0,1)$.\\
			Let $\gamma_3$ be the line joining $1 + i$ and $0$.\\
			Therefore,
			\begin{align*}
				\gamma_3(t)                  & = t + i t \\
				\therefore \dot{\gamma_3}(t) & = 1 + i
			\end{align*}
			where $t \in (0,1)$.\\
			Therefore,
			\begin{align*}
				\int\limits_{\gamma} \overline{z} \dif z & = \quad \int\limits_{0}^{1} f\left( \gamma_1(t) \right) \dot{\gamma_1}(t) \dif t                                                                                                  \\
                                                                         & \quad + \int\limits_{0}^{1} f\left( \gamma_2(t) \right) \dot{\gamma_2}(t) \dif t                                                                                                  \\
                                                                         & \quad + \int\limits_{1}^{0} f\left( \gamma_3(t) \right) \dot{\gamma_3}(t) \dif t                                                                                                  \\
                                                                         & = \int\limits_{0}^{1} \overline{t} \dif t + \int\limits_{0}^{1} \left( \overline{1 + i t} \right) i \dif t + \int\limits_{1}^{0} \left( \overline{t + i t} \right) (1 + i) \dif t \\
                                                                         & = \int\limits_{0}^{1} t \dif t + \int\limits_{0}^{1} (1 - i t) i \dif t - \int\limits_{0}^{1} (t - i t) (1 + i) \dif t                                                            \\
                                                                         & = \int\limits_{0}^{1} t \dif t + \int\limits_{0}^{1} i + t \dif t - \int\limits_{0}^{1} 2 t \dif t                                                                                \\
                                                                         & = \left. \frac{t^2}{2} \right|_{0}^{1} + \left. i t + \frac{t^2}{2} \right|_{0}^{1} - \left. t^2 \right|_{0}^{1}                                                                  \\
                                                                         & = \frac{1}{2} + i + \frac{1}{2} - 1                                                                                                                                               \\
                                                                         & = i
			\end{align*}
			Therefore, the integral of $f(z) = \overline{z}$ is non zero over a closed path.
			Hence, it cannot have an analytic primitive over $\mathbb{C}$.
	\end{enumerate}
\end{solution}

\setcounter{question}{5}
\begin{question}
	\begin{enumerate}
		\setcounter{enumi}{1}
		\item
			$\gamma$ is the triangle whose vertices are $0$, $1$, $1 + i$.
			Calculate $\displaystyle \int\limits_{\gamma} \overline{z} \dif z$.
	\end{enumerate}
\end{question}

\begin{solution}
	\begin{enumerate}[leftmargin=*]
		\item
			Let $\gamma_1$ be the line joining $0$ and $1$.\\
			Therefore,
			\begin{align*}
				\gamma_1(t)                  & = t \\
				\therefore \dot{\gamma_1}(t) & = 1
			\end{align*}
			where $t \in (0,1)$.\\
			Let $\gamma_2$ be the line joining $1$ and $1 + i$.\\
			Therefore,
			\begin{align*}
				\gamma_2(t)                  & = 1 + i t \\
				\therefore \dot{\gamma_2}(t) & = i
			\end{align*}
			where $t \in (0,1)$.\\
			Let $\gamma_3$ be the line joining $1 + i$ and $0$.\\
			Therefore,
			\begin{align*}
				\gamma_3(t)                  & = t + i t \\
				\therefore \dot{\gamma_3}(t) & = 1 + i
			\end{align*}
			where $t \in (0,1)$.\\
			Therefore,
			\begin{align*}
				\int\limits_{\gamma} \overline{z} \dif z & = \quad \int\limits_{0}^{1} f\left( \gamma_1(t) \right) \dot{\gamma_1}(t) \dif t                                                                                                  \\
                                                                         & \quad + \int\limits_{0}^{1} f\left( \gamma_2(t) \right) \dot{\gamma_2}(t) \dif t                                                                                                  \\
                                                                         & \quad + \int\limits_{1}^{0} f\left( \gamma_3(t) \right) \dot{\gamma_3}(t) \dif t                                                                                                  \\
                                                                         & = \int\limits_{0}^{1} \overline{t} \dif t + \int\limits_{0}^{1} \left( \overline{1 + i t} \right) i \dif t + \int\limits_{1}^{0} \left( \overline{t + i t} \right) (1 + i) \dif t \\
                                                                         & = \int\limits_{0}^{1} t \dif t + \int\limits_{0}^{1} (1 - i t) i \dif t - \int\limits_{0}^{1} (t - i t) (1 + i) \dif t                                                            \\
                                                                         & = \int\limits_{0}^{1} t \dif t + \int\limits_{0}^{1} i + t \dif t - \int\limits_{0}^{1} 2 t \dif t                                                                                \\
                                                                         & = \left. \frac{t^2}{2} \right|_{0}^{1} + \left. i t + \frac{t^2}{2} \right|_{0}^{1} - \left. t^2 \right|_{0}^{1}                                                                  \\
                                                                         & = \frac{1}{2} + i + \frac{1}{2} - 1                                                                                                                                               \\
                                                                         & = i
			\end{align*}
	\end{enumerate}
\end{solution}

\setcounter{question}{6}
\begin{question}
	\begin{align*}
		f(z) & = \frac{1}{2 \sqrt{z}}
	\end{align*}
	Calculate the integral over the curve starting at $1$ and ending at $9$, going over the polygon whose vertices are $1$, $1 + 8 i$, $9 + 8 i$, and then going through the right half part of a circle, with radius $4$, from $9 + 8 i$ to $9$.
\end{question}

\begin{solution}
	\begin{align*}
		f(z) & = \frac{1}{2 \sqrt{z}}
	\end{align*}
	Therefore,
	\begin{align*}
		F(z) & = \sqrt{z}
	\end{align*}
	is a primitive of $f(z)$.\\
	Let the given curve be $\gamma$.\\
	Therefore,
	\begin{align*}
		\int\limits_{\gamma} f(z) \dif z & = F(9) - F(1) \\
                                                 & = 3 - 1       \\
                                                 & = 2
	\end{align*}
\end{solution}

\end{document}
