\documentclass[fleqn, a4paper, 11pt, oneside]{amsart}
%\usepackage[top = 2cm, bottom = 1cm, left = 1cm, right = 1cm]{geometry}
\usepackage{exsheets, tasks}
\usepackage{amsmath, amssymb, amsthm} %standard AMS packages
\usepackage{marginnote} %marginnotes
\usepackage{gensymb} %miscellaneous symbols
\usepackage{commath} %differential symbols
\usepackage{xcolor} %colours
\usepackage{cancel} %cancelling terms
\usepackage[free-standing-units, space-before-unit]{siunitx} %formatting units
\usepackage{tikz, pgfplots} %diagrams
\usetikzlibrary{calc, hobby, patterns, intersections, decorations.markings}
\usepackage{graphicx} %inserting graphics
\usepackage{hyperref} %hyperlinks
\usepackage{datetime} %date and time
\usepackage{ulem} %underline for \emph{}
\usepackage{xfrac} %inline fractions
\usepackage{enumerate,enumitem} %numbered lists
\usepackage{float} %inserting floats
\usepackage{circuitikz}[american voltages, american currents] %circuit diagrams

\newcommand\numberthis{\addtocounter{equation}{1}\tag{\theequation}} %adds numbers to specific equations in non-numbered list of equations

\newcommand{\AxisRotator}[1][rotate=0]{
	\tikz [x=0.25cm,y=0.60cm,line width=.2ex,-stealth,#1] \draw (0,0) arc (-150:150:1 and 1);%
} %rotation symbols on axes

\theoremstyle{definition}
\newtheorem{example}{Example}
\newtheorem{definition}{Definition}

\theoremstyle{theorem}
\newtheorem{theorem}{Theorem}

\renewcommand{\Re}{\mathrm{Re}}
\renewcommand{\Im}{\mathrm{Im}}

\DeclareMathOperator{\Arg}{Arg}

\DeclareMathOperator{\Int}{Int}
\DeclareMathOperator{\Ext}{Ext}
\DeclareMathOperator{\boundary}{\partial}

\DeclareMathOperator{\Log}{Log}

\makeatletter
\@addtoreset{section}{part} %resets section numbers in new part
\makeatother

\renewcommand{\thesubsection}{(\arabic{subsection})}
\renewcommand{\thesection}{(\arabic{section})}

%section headings on left
\makeatletter
\def\specialsection{\@startsection{section}{1}%
	\z@{\linespacing\@plus\linespacing}{.5\linespacing}%
	%  {\normalfont\centering}}% DELETED
	{\normalfont}}% NEW
\def\section{\@startsection{section}{1}%
	\z@{.7\linespacing\@plus\linespacing}{.5\linespacing}%
	%  {\normalfont\scshape\centering}}% DELETED
	{\normalfont\scshape}}% NEW
\makeatother

%forces newline after subsection
\makeatletter
\def\subsection{\@startsection{subsection}{3}%
	\z@{.5\linespacing\@plus.7\linespacing}{.1\linespacing}%
	{\normalfont\itshape}}
\makeatother

\settasks{counter-format = tsk[1].}

\SetupExSheets{solution/print = true}

%opening
\title{Complex Functions : Assignment 5}
\author
{
	Aakash Jog\\
	ID : 989323563
}
\date{\formatdate{25}{11}{2015}}

\begin{document}

\tikzset{->-/.style={decoration={
  markings,
  mark=at position #1 with {\arrow{>}}},postaction={decorate}}}

\maketitle
%\setlength{\mathindent}{0pt}

\setcounter{question}{0}
\begin{question}
	Prove
	\begin{enumerate}
		\item $(\sin z)' = \cos z$
		\item $(\cos z)' = -\sin z$
	\end{enumerate}
\end{question}

\begin{solution}
	\begin{enumerate}[leftmargin=*]
		\item
			\begin{align*}
				\sin z & = \frac{e^{i \Arg z} - e^{-i \Arg z}}{2 i}
			\end{align*}
			Therefore,
			\begin{align*}
				(\sin z)' & = \left( \frac{e^{i \Arg z} - e^{-i \Arg z}}{2 i} \right)' \\
                                          & = \frac{i e^{i \Arg z} + i e^{i \Arg z}}{2 i}              \\
                                          & = \frac{e^{i \Arg z} + e^{i \Arg z}}{2}                    \\
                                          & = \cos z
			\end{align*}
		\item
			\begin{align*}
				\cos z & = \frac{e^{i \Arg z} + e^{-i \Arg z}}{2}
			\end{align*}
			Therefore,
			\begin{align*}
				(\cos z)' & = \left( \frac{e^{i \Arg z} + e^{-i \Arg z}}{2} \right)' \\
                                          & = \frac{i e^{i \Arg z} - i e^{-i \Arg z}}{2}             \\
                                          & = -\frac{e^{i \Arg z} + e^{i \Arg z}}{2 i}               \\
                                          & = -\sin z
			\end{align*}
	\end{enumerate}
\end{solution}

\setcounter{question}{2}
\begin{question}
	Calculate
	\begin{enumerate}
		\setcounter{enumi}{2}
		\item $\mathrm{pv}\left( (1 - i)^{4 i} \right)$
		\item $(-1)^i$
	\end{enumerate}
\end{question}

\begin{solution}
	\begin{enumerate}[leftmargin=*]
		\setcounter{enumi}{2}
		\item
			\begin{align*}
				\mathrm{pv}\left( (1 - i)^{4 i} \right) & = \Log_{-\pi}\left( (1 - i)^{4 i} \right)                                  \\
                                                                        & = \Log\left( (1 - i)^{4 i} \right)                                         \\
                                                                        & = \Log\left( e^{4 i \Log(1 - i)} \right)                                   \\
                                                                        & = \Log\left( e^{4 i \left( \ln|1 - i| + i \Arg(1 - i) \right)} \right)     \\
                                                                        & = \Log\left( e^{4 i \left( \ln \sqrt{2} - i \frac{\pi}{2} \right)} \right) \\
                                                                        & = 4 i \left( \ln \sqrt{2} - i \frac{\pi}{2} \right)                        \\
                                                                        & = 4 i \left( \frac{\ln 2}{2} - i \frac{\pi}{2} \right)                     \\
                                                                        & = 2 i \ln 2 + 2 \pi
			\end{align*}
		\item
			\begin{align*}
				(-1)^i & = i^{2 i}                                               \\
                                       & = \left( i^i \right)^2                                  \\
                                       & = \left( \left( e^{i \frac{\pi}{2}} \right)^i \right)^2 \\
                                       & = \left( e^{i^2 \frac{\pi}{2}} \right)^2                \\
                                       & = \left( e^{-\frac{\pi}{2}} \right)^2                   \\
                                       & = e^{-\pi}                                              \\
                                       & = -1
			\end{align*}
	\end{enumerate}
\end{solution}

\setcounter{question}{3}
\begin{question}
	Show that
	\begin{align*}
		\Log (1 + i)^2 & = 2 \Log(1 + i)
	\end{align*}
\end{question}

\begin{solution}
	\begin{align*}
		\Log (1 + i)^2 & = \ln\left| (1 + i)^2 \right| + i \Arg\left( (1 + i)^2 \right)     \\
                               & = \ln\left| 1 + 2 i - 1 \right| + i \Arg\left( 1 + 2 i - 1 \right) \\
                               & = \ln|2 i| + i \Arg(2 i)                                           \\
                               & = \ln 2 + i \frac{\pi}{2}                                          \\
                               & = 2 \left( \ln\left| \sqrt{2} \right| + i \frac{\pi}{4} \right)    \\
                               & = 2 \left( \ln|1 + i| + i \Arg(1 + i) \right)                      \\
                               & = 2 \Log(1 + i)
	\end{align*}
	\qed
\end{solution}

\setcounter{question}{4}
\begin{question}
	In class we showed that
	\begin{align*}
		\dod{\Log z}{z} & = \frac{1}{z}
	\end{align*}
	using the polar Cauchy-Riemann equations.
	Prove it again using the chain rule and the fact that $z = e^{\Log z}$.
\end{question}

\begin{solution}
	\begin{align*}
		\dod{e^{\Log z}}{z}        & = \dod{e^{\Log z}}{\Log z} \dod{\Log z}{z} \\
		\therefore \dod{z}{z}      & = e^{\Log z} \dod{\Log z}{z}               \\
		\therefore 1               & = e^{\Log z} \dod{\Log z}{z}               \\
                                           & = z \dod{\Log z}{z}                        \\
		\therefore \dod{\Log z}{z} & = \frac{1}{z}
	\end{align*}
	\qed
\end{solution}

\end{document}
