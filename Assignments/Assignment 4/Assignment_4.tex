\documentclass[fleqn, a4paper, 11pt, oneside]{amsart}
%\usepackage[top = 2cm, bottom = 1cm, left = 1cm, right = 1cm]{geometry}
\usepackage{exsheets, tasks}
\usepackage{amsmath, amssymb, amsthm} %standard AMS packages
\usepackage{marginnote} %marginnotes
\usepackage{gensymb} %miscellaneous symbols
\usepackage{commath} %differential symbols
\usepackage{xcolor} %colours
\usepackage{cancel} %cancelling terms
\usepackage[free-standing-units, space-before-unit]{siunitx} %formatting units
\usepackage{tikz, pgfplots} %diagrams
\usetikzlibrary{calc, hobby, patterns, intersections, decorations.markings}
\usepackage{graphicx} %inserting graphics
\usepackage{hyperref} %hyperlinks
\usepackage{datetime} %date and time
\usepackage{ulem} %underline for \emph{}
\usepackage{xfrac} %inline fractions
\usepackage{enumerate,enumitem} %numbered lists
\usepackage{float} %inserting floats
\usepackage{circuitikz}[american voltages, american currents] %circuit diagrams

\newcommand\numberthis{\addtocounter{equation}{1}\tag{\theequation}} %adds numbers to specific equations in non-numbered list of equations

\newcommand{\AxisRotator}[1][rotate=0]{
	\tikz [x=0.25cm,y=0.60cm,line width=.2ex,-stealth,#1] \draw (0,0) arc (-150:150:1 and 1);%
} %rotation symbols on axes

\theoremstyle{definition}
\newtheorem{example}{Example}
\newtheorem{definition}{Definition}

\theoremstyle{theorem}
\newtheorem{theorem}{Theorem}

\newcommand{\curl}{\mathrm{curl\,}}

\newcommand{\Arg}{\mathrm{Arg}}

\newcommand{\Int}{\mathrm{Int}}
\newcommand{\Ext}{\mathrm{Ext}}
\newcommand{\boundary}{\partial}

\makeatletter
\@addtoreset{section}{part} %resets section numbers in new part
\makeatother

\renewcommand{\thesubsection}{(\arabic{subsection})}
\renewcommand{\thesection}{(\arabic{section})}

%section headings on left
\makeatletter
\def\specialsection{\@startsection{section}{1}%
	\z@{\linespacing\@plus\linespacing}{.5\linespacing}%
	%  {\normalfont\centering}}% DELETED
	{\normalfont}}% NEW
\def\section{\@startsection{section}{1}%
	\z@{.7\linespacing\@plus\linespacing}{.5\linespacing}%
	%  {\normalfont\scshape\centering}}% DELETED
	{\normalfont\scshape}}% NEW
\makeatother

%forces newline after subsection
\makeatletter
\def\subsection{\@startsection{subsection}{3}%
	\z@{.5\linespacing\@plus.7\linespacing}{.1\linespacing}%
	{\normalfont\itshape}}
\makeatother

\settasks{counter-format = tsk[1].}

\SetupExSheets{solution/print = true}

%opening
\title{Complex Functions : Assignment 4}
\author
{
	Aakash Jog\\
	ID : 989323563
}
\date{\formatdate{18}{11}{2015}}

\begin{document}

\tikzset{->-/.style={decoration={
  markings,
  mark=at position #1 with {\arrow{>}}},postaction={decorate}}}

\maketitle
%\setlength{\mathindent}{0pt}

\setcounter{question}{1}
\begin{question}
	Prove that the following functions are harmonic and find their conjugates.
	\begin{enumerate}
		\setcounter{enumi}{1}
		\item $u(x,y) = e^{-x} (x \sin y - y \cos y)$
	\end{enumerate}
\end{question}

\begin{solution}
	\begin{enumerate}
		\setcounter{enumi}{1}
		\item
			\begin{align*}
				u(x,y) & = e^{-x} (x \sin y - y \cos y)
			\end{align*}
			Therefore,
			\begin{align*}
				u_x                & = -e^{-x} (x \sin y - y \cos y) + e^{-x} (\sin y)                \\
				\therefore u_{x x} & = e^{-x} (x \sin y - y \cos y) - e^{-x} (\sin y) - e^{-x} \sin y \\
                                                   & = e^{-x} \left( x \sin y - y \cos y - 2 \sin y \right)           \\
				u_y                & = e^{-x} (x \cos y - \cos y + y \sin y)                          \\
				u_{y y}            & = e^{-x} \left( -x \sin y + \sin y + \sin y + y \cos y \right)   \\
                                                   & = e^{-x} \left( -x \sin y + 2 \sin y + y \cos y \right)
			\end{align*}
			Therefore,
			\begin{align*}
				u_{x x} + u_{y y} & = 0
			\end{align*}
			Therefore, the function is harmonic.
	\end{enumerate}
\end{solution}

\setcounter{question}{2}
\begin{question}
	Express $\Re(\sin z)$ and $\Im(\sin z)$ as real functions dependent on $x$ and $y$ where $z = x + i y$.
	Simplify your answer using the definitions of the hyperbolic trigonometric functions.
\end{question}

\begin{solution}
	\begin{align*}
		\Re(\sin z) & = \Re\left( \frac{e^{i z} - e^{-i z}}{2 i} \right)                                                        \\
                            & = \Re\left( \frac{e^{i (x + i y)} - e^{-i (x + i y)}}{2 i} \right)                                        \\
                            & = \Re\left( \frac{e^{i x - y} - e^{-i x + y}}{2 i} \right)                                                \\
                            & = \Re\left( \frac{e^{i x} e^{-y} - e^{-i x} e^{y}}{2 i} \right)                                           \\
                            & = \Re\left( \frac{e^{-y} \cos x + i e^{-y} \sin x - e^y \cos(-x) - i e^y \sin(-x)}{2 i} \right)           \\
                            & = \Re\left( \frac{e^{-y} \cos x - e^y \cos(-x)}{2 i} + i \frac{e^{-y} \sin x - e^y \sin(-x)}{2 i} \right) \\
                            & = \Re\left( \frac{e^{-y} \sin x + e^y \sin x}{2} - i \frac{e^{-y} \cos x - e^y \cos x}{2} \right)         \\
                            & = \frac{e^{-y} \sin x + e^y \sin x}{2}                                                                    \\
                            & = \sin x \left( \frac{e^{-y} + e^y}{2} \right)                                                            \\
                            & = \sin x \cosh y
	\end{align*}
	\begin{align*}
		\Im(\sin z) & = \Im\left( \frac{e^{i z} - e^{-i z}}{2 i} \right)                                                        \\
                            & = \Im\left( \frac{e^{i (x + i y)} - e^{-i (x + i y)}}{2 i} \right)                                        \\
                            & = \Im\left( \frac{e^{i x - y} - e^{-i x + y}}{2 i} \right)                                                \\
                            & = \Im\left( \frac{e^{i x} e^{-y} - e^{-i x} e^{y}}{2 i} \right)                                           \\
                            & = \Im\left( \frac{e^{-y} \cos x + i e^{-y} \sin x - e^y \cos(-x) - i e^y \sin(-x)}{2 i} \right)           \\
                            & = \Im\left( \frac{e^{-y} \cos x - e^y \cos(-x)}{2 i} + i \frac{e^{-y} \sin x - e^y \sin(-x)}{2 i} \right) \\
                            & = \Im\left( \frac{e^{-y} \sin x + e^y \sin x}{2} - i \frac{e^{-y} \cos x - e^y \cos x}{2} \right)         \\
                            & = \frac{e^y \cos x - e^{-y} \cos x}{2}                                                                    \\
                            & = \cos x \frac{e^y - e^{-y}}{2}                                                                           \\
                            & = \cos x \sinh y
	\end{align*}
\end{solution}

\setcounter{question}{3}
\begin{question}
	Find the image of the strip $|y| < \pi$ under the map $f(z) = e^z$.
\end{question}

\begin{solution}
	\begin{align*}
		f(z) & = e^z         \\
                     & = e^{x + i y} \\
                     & = e^x e^{i y}
	\end{align*}
	Therefore, as $|y| < \pi$,
	\begin{gather*}
		-\pi < y < \pi\\
		\therefore e^{-i \pi} < e^{i y} < e^{i \pi}
	\end{gather*}
	Therefore, the image is a disk with radius $e^x$, except for the negative real axis.\\
	Therefore, as $x \in (-\infty,\infty)$, the image of the strip is $\mathbb{C} \setminus \mathbb{R}^-$.
\end{solution}

\setcounter{question}{5}
\begin{question}
	Prove the following identities.
	\begin{enumerate}
		\setcounter{enumi}{0}
		\item $\cos^2 z + \sin^2 z = 1$
		\item $\sin z \cos w = \frac{1}{2} \left( \sin(z + w) + \sin(z - w) \right)$
	\end{enumerate}
\end{question}

\begin{solution}
	\begin{enumerate}[leftmargin=*]
		\setcounter{enumi}{0}
		\item
			\begin{align*}
				\cos^2 z + \sin^2 z & = \left( \frac{e^{i z} + e^{-i z}}{2} \right)^2 + \left( \frac{e^{i z} - e^{-i z}}{2 i} \right)^2 \\
                                                    & = \frac{e^{2 i z} + 2 + e^{-2 i z}}{4} - \frac{e^{2 i z} - 2 + e^{-2 i z}}{4}                     \\
                                                    & = \frac{4}{4}                                                                                     \\
                                                    & = 1
			\end{align*}
			\qed
			\begin{align*}
				\sin z \cos w & = \left( \frac{e^{i z} - e^{-i z}}{2 i} \right) \left( \frac{e^{i w} + e^{-i w}}{2} \right)                          \\
                                              & = \frac{e^{i z + i w} + e^{i z - i w} - e^{-i z + i w} - e^{-i z - i w}}{4 i}                                        \\
                                              & = \frac{1}{2} \left( \frac{e^{i z + i w} - e^{-i z - i w}}{2 i} + \frac{e^{i z - i w} - e^{-i z + i w}}{2 i} \right) \\
                                              & = \frac{1}{2} \left( \sin(z + w) + \sin(z - w) \right)
			\end{align*}
	\end{enumerate}
\end{solution}

\end{document}
