\documentclass[fleqn, a4paper, 11pt, oneside]{amsart}
%\usepackage[top = 2cm, bottom = 1cm, left = 1cm, right = 1cm]{geometry}
\usepackage{exsheets, tasks}
\usepackage{amsmath, amssymb, amsthm} %standard AMS packages
\usepackage{marginnote} %marginnotes
\usepackage{gensymb} %miscellaneous symbols
\usepackage{commath} %differential symbols
\usepackage{xcolor} %colours
\usepackage{cancel} %cancelling terms
\usepackage[free-standing-units, space-before-unit]{siunitx} %formatting units
\usepackage{tikz, pgfplots} %diagrams
\usetikzlibrary{calc, hobby, patterns, intersections, decorations.markings}
\usepackage{graphicx} %inserting graphics
\usepackage{hyperref} %hyperlinks
\usepackage{datetime} %date and time
\usepackage{ulem} %underline for \emph{}
\usepackage{xfrac} %inline fractions
\usepackage{enumerate,enumitem} %numbered lists
\usepackage{float} %inserting floats
\usepackage{circuitikz}[american voltages, american currents] %circuit diagrams

\newcommand\numberthis{\addtocounter{equation}{1}\tag{\theequation}} %adds numbers to specific equations in non-numbered list of equations

\newcommand{\AxisRotator}[1][rotate=0]{
	\tikz [x=0.25cm,y=0.60cm,line width=.2ex,-stealth,#1] \draw (0,0) arc (-150:150:1 and 1);%
} %rotation symbols on axes

\theoremstyle{definition}
\newtheorem{example}{Example}
\newtheorem{definition}{Definition}

\theoremstyle{theorem}
\newtheorem{theorem}{Theorem}

\renewcommand{\Re}{\mathrm{Re}}
\renewcommand{\Im}{\mathrm{Im}}

\DeclareMathOperator{\Arg}{Arg}

\DeclareMathOperator{\Int}{Int}
\DeclareMathOperator{\Ext}{Ext}
\DeclareMathOperator{\boundary}{\partial}

\DeclareMathOperator{\Log}{Log}
\DeclareMathOperator{\pv}{pv}

\DeclareMathOperator{\length}{length}

\makeatletter
\@addtoreset{section}{part} %resets section numbers in new part
\makeatother

\renewcommand{\thesubsection}{(\arabic{subsection})}
\renewcommand{\thesection}{(\arabic{section})}

%section headings on left
\makeatletter
\def\specialsection{\@startsection{section}{1}%
	\z@{\linespacing\@plus\linespacing}{.5\linespacing}%
	%  {\normalfont\centering}}% DELETED
	{\normalfont}}% NEW
\def\section{\@startsection{section}{1}%
	\z@{.7\linespacing\@plus\linespacing}{.5\linespacing}%
	%  {\normalfont\scshape\centering}}% DELETED
	{\normalfont\scshape}}% NEW
\makeatother

%forces newline after subsection
\makeatletter
\def\subsection{\@startsection{subsection}{3}%
	\z@{.5\linespacing\@plus.7\linespacing}{.1\linespacing}%
	{\normalfont\itshape}}
\makeatother

\settasks{counter-format = tsk[1].}

\SetupExSheets{solution/print = true}

%opening
\title{Complex Functions : Assignment 9}
\author
{
	Aakash Jog\\
	ID : 989323563
}
\date{\formatdate{30}{12}{2015}}

\begin{document}

\tikzset{->-/.style={decoration={
  markings,
  mark=at position #1 with {\arrow{>}}},postaction={decorate}}}

\maketitle
%\setlength{\mathindent}{0pt}

\setcounter{question}{2}
\begin{question}
	In this question $f$ is an entire function.
	You need to prove that under the following conditions, for each condition separately, it must be constant.
	\begin{enumerate}
		\item For every $z \in \mathbb{C}$, $\Re(f) \ge 0$.
		\item For every $z \in \mathbb{C}$, $\left| f(z) \right| \neq 1$.
		\item For every $z \in \mathbb{C}$, $f(z) \notin [0,1]$.
	\end{enumerate}
\end{question}

\begin{solution}
	\begin{enumerate}[leftmargin=*]
		\item
			\begin{align*}
				\Re(f(z)) &\ge 0\\
				\therefore \Re(f(z) + 1) &\ge 1\\
				\therefore \left| f(z) + 1 \right| &\ge 1\\
				\therefore 1 &\ge \frac{1}{\left| f(z) + 1 \right|}
			\end{align*}
			Therefore, as $\frac{1}{f(z) + 1}$ is entire, and is bounded by $1$, by Liouville's theorem, it is constant.
			Hence, $f(z)$ is also constant.
		\item
		\item
			Let
			\begin{align*}
				g(z) &= \frac{1}{f(z)}
			\end{align*}
			Therefore, $g(z)$ is entire, and bounded by $1$.
			Therefore, by Liouville's theorem, $g(z)$ is constant.
			Hence, $f(z)$ is constant.
	\end{enumerate}
\end{solution}

\setcounter{question}{3}
\begin{question}
	Calculate $\displaystyle \int\limits_{0}^{2 \pi} \frac{\dif t}{a \cos t + b \sin t + c}$, where $a$, $b$, $c$ satisfy $\sqrt{a^2 + b^2} = 1 < c$.
\end{question}

\begin{solution}
	Let
	\begin{align*}
		a &= \cos \alpha\\
		b &= \sin \alpha
	\end{align*}
	Therefore,
	\begin{align*}
		a \cos t + b \sin t &= \cos \alpha \cos t + \sin \alpha \sin t\\
		&= \cos(\alpha - t)
	\end{align*}
	Therefore,
	\begin{align*}
		\int\limits_{0}^{2 \pi} \frac{\dif t}{\cos(\alpha - t) + c} &= \int\limits_{0}^{2 \pi} \frac{\dif t}{\frac{e^{i (\alpha - t)} + e^{i (t - \alpha)}}{2} + c}
	\end{align*}
	Let
	\begin{align*}
		z &= i (\alpha - t)\\
		\therefore \dif z &= -i \dif t\\
		\therefore \dif t &= \frac{\dif z}{-i}\\
		&= i \dif z
	\end{align*}
	Therefore,
	\begin{align*}
		\int\limits_{0}^{2 \pi} \frac{\dif t}{\cos(\alpha - t) + c} &= \int\limits_{i \alpha}^{i \alpha - 2 \pi i} \frac{i \dif z}{\frac{e^{i z} + e^{-i z}}{2} + c}\\
		&= \int\limits_{i \alpha}^{i \alpha - 2 \pi i} \frac{2 i \dif z}{e^{i z} + e^{-i z} + 2 c}\\
		&= \left. \frac{2 \tan ^{-1}\left( \frac{c + e^{i z}}{\sqrt{1 - c^2}}\right)}{\sqrt{1 - c^2}} \right|_{i \alpha}^{i \alpha - 2 \pi i}\\
		&= \frac{2 \tan^{-1} \left( \frac{c + e^{i \alpha} e^{-2 \pi i}}{\sqrt{1 - c^2}} \right)}{\sqrt{1 - c^2}} - \frac{2 \tan^{-1} \left( \frac{c + e^{i \alpha}}{\sqrt{1 - c^2}} \right)}{\sqrt{1 - c^2}}\\
		&= \frac{2 \tan^{-1} \left( \frac{c + e^{i \alpha} }{\sqrt{1 - c^2}} \right)}{\sqrt{1 - c^2}} - \frac{2 \tan^{-1} \left( \frac{c + e^{i \alpha}}{\sqrt{1 - c^2}} \right)}{\sqrt{1 - c^2}}\\
		&= 0
	\end{align*}
\end{solution}

\setcounter{question}{4}
\begin{question}
	If $p(z)$ is a polynomial of degree $n \ge 1$, and there exists $\alpha > 0$, such that
	\begin{align*}
		\left| p(z) \right| &\le \alpha |z|
	\end{align*}
	Then, there exists $c \in \mathbb{C}$, such that
	\begin{align*}
		p(z) &= c z
	\end{align*}
\end{question}

\begin{solution}
	\begin{align*}
		\left| p(z) \right| &\le \alpha |z|
	\end{align*}
	Therefore,
	\begin{align*}
		\Re\left( p(z) \right) &\le \alpha \Re(z)\\
	\end{align*}
	Similarly,
	\begin{align*}
		\Im\left( p(z) \right) &\le \alpha \Im(z)\\
	\end{align*}
	Therefore,
	\begin{align*}
		p(z) &\le \Re(z) + i \Im(z)
	\end{align*}
	Therefore, $\exists c \in \mathbb{C}$, such that
	\begin{align*}
		p(z) &= c z
	\end{align*}
\end{solution}

\end{document}
